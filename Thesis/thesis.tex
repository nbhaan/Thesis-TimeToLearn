 \documentclass[oneside,a4paper,12pt]{book}
%\pagestyle{headings}
\frontmatter
\input{preamble}

% A B S T R A C T
% % % % % % % % % % % % % % % % % % % % % % % % % % % % % % % % % %
\chapter*{\centering Abstract}
\begin{quotation}
\noindent 

Learning has changed over the years. People do not learn only from books anymore. Although Books are still used, other tools for learning are gaining popularity. Websites that can test the knowledge about words of another language of the user are quit common and they have improved themselves over the years.

There is however a disadvantage. The user still needs to find the time to sit down in front of the screen and focus to learn the words. With the upcoming market of smartwatches a new possibility for learning is created. A possibility that makes the user able to learn words whenever the user wants to. 

The smartwatch is however a relatively new platform with a small screen and thus we had to investigate what is the best way to build an app for the smartwatch. And can an app for the smartwatch help to accelerate the memorization process of somebody who is learning the vocabulary of a second language?

The app is useful for busy people who can read and understand the language, but want to improve their vocabulary by reading articles in this language and selecting the words the user did not know. These words can be learned by the smartwatch app to improve their vocabulary. 
The smartwatch is a suitable solutation, because it is quickly accessible which makes it ideal for situations like waiting for and waiting in the elevator or waiting for the bus or other situations where time is lost by waiting. The user only has to twist his or her wrist to continue learning. This is the reason why it is more convenient to use than an app for the smartphone. It is much more easier to access the app on the watch than on a smartphone.

Applications that are built with the same motivation as our app are mostly built for a smartphone like an Android wallpaper. Our approach is different, because we are going to try to use a smartwatch.

Several users have used the app for five days and they provided us with results and feedback for our research.

\end{quotation}
\clearpage


% C O N T E N T S 
% % % % % % % % % % % % % % % % % % % % % % % % % % % % % % % % % % % % % % % %
\tableofcontents

\mainmatter
%%%%%%%%%%%%%%%%%%%%%%%%%%%%%%%%%%
%%%% NEW CHAPTER %%%%%%%%%%%%%%%%%%%%%
%%%%%%%%%%%%%%%%%%%%%%%%%%%%%%%%%%
\chapter{Introduction}
\label{cha:introduction}
The way people learn has changed over the years. The development of information and communication technology has lead to new ways of learning. People can now find fast and really specific information on the web and learn new things using ICT tools, like laptops, iPads and other electronic devices. The traditional book is being less and less used. We can see this in high schools where students learn their languages on the laptop with a rehearse program. These rehearsing programs should accelerate the memorization process, so the student has more time for other stuff and can work more efficiently. 

Accelerating the memorization process of somebody who is learning the vocabulary of a second language is not only valuable for students, but also for other people who want to learn a second language for other reasons. But even with rehearsing programs learning a second language can be hard and time consuming thing to do. Since learning a second language is so time consuming, people often don't have enough time for it. They are busy with other important things like work and don't have the time to really learn every day. Although for these type of people it seems like they have no time to learn, but in fact the time they lose for waiting for the bus, taking the elevator, walking to the car could be used more efficiently. Using these little time units is called micro learning. In this thesis we are going to try to find out how we can fill these little time spots in the best way using a smartwatch.

The application for the smartwatch will be an watch face, this is the screen that user will see when the screen is on. The main function of the watch face is to show time, our application however offers the user a word which can be revealed by tapping on the screen. The user will be stimulated to think of the right translation before seeing it, after revealing it the user can give the watch feedback by pressing the green ('I had it right in my head') or red button ('I had it wrong in my head'). The algorithm used for displaying words is based on the way in which people learn with flashcards. This is to speed up the memorization process, wrong words will be repeated earlier then words which the user had right in his mind.

The app for the smart watch is part of Zeeguu, which is a research project designed to speed \& fun up vocabulary learning in a new language. Its based on three fundamental principles: only read the stuff you like, have words everywhere with you and practice with personalized exercises. This basically means the users read articles they like, tap on the words they don't know and practice them later. The words will be saved in their accounts and can be accessed at any time. We build an app for the smartwatch, so users have an complementary tool for learning their words in the Zeeguu account and thus increase the memorization process of learning a second new language. In thesis we will research what is the best way to build such an app, therefore an user study was designed to answer the following research question: 

What is the best way to build an app for the smartwatch to accelerate the memorization process of somebody who is learning the vocabulary of a second language?

\paragraph{Structure of this Thesis}
\begin{itemize}
\item Related work: the general research field is described and how others tried to solve this solve problem.
\item The design: the user interface is described, explaining the choices which have been made.
\item The implementation: the code structure, algorithms and implementation choices are explained.
\item Usage results: explaining the tests and a summary of the results with various diagrams.
\item Evaluation: a discussion about why this research was successful.
\item Conclusion and future work: conclusions about the usage results and describing possible future work.
\end{itemize}



\chapter {Related Work}
Describe the field in general and how others have tried to solve this problem.

Your goals are to:
\begin{itemize}
	\item show you are aware of current state of knowledge
(theoretical, methodological, applied) that relates to your
research topic
	\item To indicate a gap/question worthy of investigation 
\end{itemize}




\chapter{The Problem Statement}
describe in detail the problem you are trying to solve.

\chapter {The Approach}
describe your approach to solving the problem. Describe any potential weaknesses of your approach.

\chapter {The Implementation}
describe how you implemented your approach. If it is a software system give diagrams, relevant algorithms etc.

packages that you can look at for code formatting: 

- http://mirror.unl.edu/ctan/macros/latex/contrib/minted/minted.pdf
- https://en.wikibooks.org/wiki/LaTeX/Source\_Code\_Listings


\chapter {The Evaluation}
describe how you evaluated to show that your approach was successful. You may need a methods section, a results section and a conclusion section.


\chapter {Conclusion and Future Work}
summarize your thesis again as in the introduction. Describe how your evaluation revealed that your system is successful. Describe future work in this area.



%END Doc
%-------------------------------------------------------

\bibliography{thesis}
\bibliographystyle{plain}

\end{document}